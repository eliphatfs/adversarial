\documentclass{article}
\usepackage{homework}
\usepackage{tikz}
\usetikzlibrary{positioning}
\usetikzlibrary{arrows}
\usepackage[ruled, vlined]{algorithm2e} % 算法与伪代码

\newtheorem{fact}{Fact}

\title{Baseline is All You Need: the Story of Failing to Outperform Project Baselines}
\author{RealEEMan \quad RealEEMan's 大腿挂件}
\date{}

\begin{document}
\maketitle
\section{Introduction}\label{Sec:Intro}
In this work, we detail our struggle of trying and eventually failing to do better than the baselines provided by our TA.

\section{Related Work}\label{Sec:RelatedWork}
Most of these work did far better than us. Their works were based on comprehensive research into adversarial models and sound mathematical 

\section{Frank-Wolfe Defenses}\label{Sec:DefenseMethodology}
We had little ideas on adversarial defenses.

At the beginning, we wanted to exploit Variational Auto Encoders (VAE) to enhance input continuity, and hoped that a more continuous input space would boost model robustness. Unfortunately, training a VAE turned out to be one of the world's best time-wasting torture. To the best of our knowledge, it was second only to compiling and running Kaldi. Worse still, the not-properly-trained VAE refused to produce anything useful, so we got stuck.

\section{Results}\label{Sec:Results}
Let me see what my code outputs.

\verb|TypeError: NoneType is not callable.|

\verb|MemoryError: Unable to allocate 11.3 TiB|

\verb|Process killed.|

\verb|#15: Initializing libiomp5md.dll, but found libiomp5md.dll already initialized.|

\section{Conclusion}\label{Sec:Conclusion}
\begin{fact}[Fundamental Theorem of AI2612]
    \label{Fact:FundamentalFactOfAI2612}
    Baseline is all you need.
\end{fact}
It follows from Fact \ref{Fact:FundamentalFactOfAI2612} immediately that the baselines are the real state-of-the-art's, and that this project can be done by simply cloning a baseline. We are weak.

\end{document}